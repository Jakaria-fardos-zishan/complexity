% ****** Start of file apssamp.tex ******
%
%   This file is part of the APS files in the REVTeX 4.2 distribution.
%   Version 4.2a of REVTeX, December 2014
%
%   Copyright (c) 2014 The American Physical Society.
%
%   See the REVTeX 4 README file for restrictions and more information.
%
\documentclass[%
reprint,
amsmath,amssymb,
aps,
floatfix,
]{revtex4-2}

\usepackage{graphicx}
\usepackage{dcolumn}
\usepackage{bm}
\usepackage{natbib}
\usepackage{subcaption}
\usepackage{placeins}
\usepackage{amsmath}
\usepackage{booktabs}
\begin{document}
	
	\preprint{APS/123-QED}
	
	\title{Quantitative analysis of the complexity of dynamical systems}

	
	\author{Golam Dastegir Al-Quaderi}
	\altaffiliation[Also at ]{Physics Department, University of Dhaka.}
	\email{dastegir@du.ac.bd}
	\author{Al-amin Wazed Shourov}
	\email{wazed2909@gmail.com}

	\author{Tahrim Ahmed}
	\email{tahrimahmed8888@gmail.com}
	\author{Jakaria Fardos Zishan}
	\email{fardos.zishan01@gmail.com}	
		\affiliation{%
	University of Dhaka
	}

	
	\begin{abstract}
		The Lorenz attractor is an example for studying deterministic chaos and complex dynamics. In this work, we performed an analysis of the attractor by combining information-theoretic measures with dynamics. Specially, we computed Shannon entropy, Renyi entropy, permutation entropy, weighted permutation entropy, mutual information etc. In parallel we tried to understand the geometric properties of this system. Our analysis showed a strong relation between these measurements and dynamical activities. The result revealed that these quantities can be used to study dynamical configurations. The integration of these approaches open up a consistent characterization of Lorenz dynamics, providing a powerful blueprint for analysis chaotic system on a wide range.
		
	\end{abstract}
	
	\maketitle
	
	\section{\label{sec:level1}Introduction}
	The Lorenz attractor~\cite{lorenz2017deterministic} arises from a simplified model of atmospheric convection developed by Edward N. Lorenz in 1963. It is also studied in information theory, complexity science, and geometric analysis.\\
	It is given by three coupled differential equations. The solution to these equations traces a beautiful butterfly-shaped trajectory in xyz-space. By simply observing this trajectory, we recognize that it is not a simple system. Yet quantifying this complexity remains difficult. There are some defined ways to measure complexity such as Kolmogorov complexity\cite{kolmogorov1965three}, López-Ruiz-Mancini-Calbet (LMC) complexity\cite{lopez1995statistical} etc.\\
	We can compute existing complexity measures, but this immediately raises a vital question: "Are these measures even sufficient for capturing the true complexity of this system? If not, what alternative approach can we discover?"\\
	If we succeed in establishing a meaningful new quantity for measuring complexity, it must carry genuine physical significance.\\
	Simultaneously, we can explore other measurements that can describe the system like entropy, as entropy in general quantifies the unpredictability and the lack of information. One finds several entropies such as Boltzmann entropy, Shannon entropy \cite{shannon1948mathematical}, Kolmogorov–Sinai entropy (KS entropy), approximate entropy \cite{pincus1991approximate}, permutation entropy\cite{bandt2002permutation}, and many more. With so many entropy definitions available, we must ask: Do all entropies tell the same story, or do they reveal fundamentally different aspects? Moreover, we could seek novel kinds of entropy that might describe this system more effectively.\\
	Furthermore, the Lorenz attractor is an example of deterministic chaos. The term deterministic chaos means that the system originates from a set of deterministic equations, yet produces behavior that is unpredictable in the long term although it is predictable in the short term. Given its chaotic nature, we might also calculate quantities like Lyapunov exponents\cite{wolf1985determining}, which explain how sensitive a dynamical system is to initial conditions etc.\\
	Beyond dynamics, we may also look for geometric properties – extracted not only from the trajectory itself, but also from phase space, velocity space, and beyond. (need to write what kind of geometry we may find)\\
	In order to calculate all these, we have three time series (x, y, z). One may study them statistically or using information theory. In either case, each time series can be studied individually, but we must understand their interdependence. Therefore, these series demand mutual analysis.\\
	To uncover relationships between key quantities, we can employ tools like mutual information\cite{kraskov2004estimating}, transfer entropy\cite{PhysRevLett.85.461}, etc. Even though the system is chaotic, we can search for periodicity using autocorrelation\cite{yule1926nonsense}, which measures how a signal relates to itself over time, and the Higuchi Fractal Dimension\cite{HIGUCHI1988277} for estimating fractal dimension. Moreover, we need to understand how we can find geometric properties from these time series.\\
	In order to understand it, we could try to find if there is any conserved quantity like energy, work, etc. However, as it is a bound orbit, we might consider whether there exist any homoclinic or heteroclinic orbits. If they exist, how many of them are there?

	\section{Literature Survey}
	\subsection{Shannon Entropy}
	Shannon entropy (SE), introduced by Claude Shannon (site), quantifies the average uncertainty of a random variable. For a discrete variable $X$ with outcomes ${x_i}$, it is defined as:
	
	\begin{equation}
		H(X) = -\sum_{i} p(x_i) \log_2 p(x_i)
		\label{eq:shannon}
	\end{equation}
	where $p(x_i)$ is the probability of outcome $x_i$.where,
	\[
	H(X) \geq 0
	\]
	SE calculates the entropy of symbolic sequences. High SE indicates high chaoticity. However, SE ignores geometric structure. As a result, the same SE values can result from geometrically different configurations. Besides, it cannot capture directional information flow.\\
	\subsection{Kolmogorov-Sinai Entropy}
	The Kolmogorov-Sinai (KS) entropy, introduced independently by Kolmogorov\cite{kolmogorov1958new} (1958) and Sinai\cite{sinai1959notion} (1959). It measure the rate of information production in deterministic dynamical systems which refers to the rate at which a system generates uncertainty about its future state. KS entropy is defined as:
	\begin{equation}
		h_u  = \operatorname*{sup}_{\mathcal{P}} \operatorname*{\lim}_{n \to \infty} \frac{1}{n} h(P_n)	
	\end{equation}
	\subsection{Rényi entropy}
	Rényi entropy\cite{renyi1961measures}  is a generalization of Shannon entropy designed to measure the uncertainty or randomness of a system using a parameter $\alpha$. For a discrete probability distribution $P = (p_1,p_2,...,p_k)$ it is defined as:
	\begin{equation}
		H_\alpha (P) = \frac{1}{1 - \alpha} \log (\sum_{i=1}^{k} p_{i}^{\alpha})
	\end{equation}
	R\'enyi entropy is particularly useful in analyzing contexts where a single exponent may not suffice. 
	here:\\
	$\alpha \geq 0$\\ 
	$\alpha \neq 1$ (for $\alpha$ = 1, it converge to Shannon entropy)\\
	\subsection{Transfer Entropy}
	Transfer entropy (TE) quantifies the directional flow of information between two time series. It is widely used in chaos theory, neuroscience and complex system making it highly relevant for studying Lorenz attractor. For two time series $X$ and $Y$ the transfer entropy from Y to X is:
	\begin{equation}
		T_{Y\to X} = \sum p(x_{t+1}, x_{t}^{(k)}, y_{t}^{(k)}) \log\left( \frac{p(x_{t+1} \mid x_{t}^{(k)}, y_{t}^{(k)})}{p(x_{t+1} \mid x_{t}^{(k)})} \right)
	\end{equation}
	TE can identify dominant driving variable. It also detects dynamic causation. Besides Lyapunov exponents measure local instability and KS entropy calculates global unpredictability but not coupling structure but TE complements these by quantifying information transfer pathway.
	\subsection{Permutation Entropy}
	Permutation entropy (PE), introduced by Bandt and Pompe in 2002, is a simple method for quantifying complexity. Unlike traditional entropy PE calculates the unpredictability or randomness of a signal by analyzing the ordinal patterns of its values. PE is broadly use is the study of chaotic system.
	\begin{equation}
		H_{PE} = -\sum_{i} P(\pi_i) \log P(\pi_i)
	\end{equation}
	Here $P(\pi_i)$ is the probability of pattern $\pi_i$.\\
	High PE refers irregular behavior and low PE indicates deterministic nature. PE ignores amplitude as a result weighted permutation entropy is used. Moreover, it can not handle equal values is neighborhood. For this reason we can use weighted permutation entropy.
	\subsection{Weighted Permutation Entropy}
	Weighted Permutation Entropy\cite{fadlallah2013weighted} (WPE) is an advanced variant of PE. Unlike PE it includes amplitude information into complexity analysis. WPE is a powerful tool for chaotic system like Lorenz attractor as both ordinal patterns and amplitude variations play critical role here.
	\begin{equation} 
		H_{WPE} = -\sum_{\pi} p_w(\pi) \ln p_w (\pi)
	\end{equation} 
	
	Where $p_w(\pi)$ is weighted probability of observing an ordinal pattern $\pi$ in a time series.\\
	(need to make sure the eq for wpe)\\
	where $w_i$ is variance $w_i = \frac{1}{m} \sum_{k=1}^{m} (x_{i+(k-1)\tau} - \overline{x}_i) $
	\subsection{Mutual Information}
	Mutual Information (MI) measures the nonlinear dependence between two random variables. It estimates how much knowing one variable reduces uncertainty about another. In the context of dynamical systems , it's used to study interdependence between different time series or system components. Let $X$ and $Y$ be two random variable and $p(x,y)$ ber joint probability and $p(x)$ and $p(y)$ be marginal probability. In that case MI is defined as:
	\begin{equation}
		I(X,Y) = \sum_{x\in X} \sum_{y\in Y} p(x,y) \log \frac{p(x,y)}{p(x)p(y)}
	\end{equation}
	It can also be interpreted as:
	\begin{equation}
		I(X,Y) = H(X) + H(Y) - H(X,Y)
	\end{equation}
	where $H(X)$, $H(Y)$ and $H(X,Y)$ and Shannon entropy of $X$, $Y$ and joint SE of $X$ and $Y$ respectively.
	\subsection{Higuchi Fractal Dimension}
	Higuchi Fractal Dimension (HFD) is an efficient way to estimate fractal dimention (FD)\cite{mandelbrot1982fractal} of a time series. It was introduced by T. Higuchi (1988). It gives a quantitative measure of complexity. Besides chaos analysis it is widely used in neuroscience and signal processing.\\
	need to finish it
	\subsection{Autocorrelation}
	Autocorrelation computes how much a signal correlates with a time-shifted or delayed version of  itself. It can be used for detecting periodicity and predictability by evaluating how its past values influence the future values of a time series.\\
	For a time series ${x_t}_{t=1}^N$ and lag $\tau$, autocorrelation $R(\tau)$ is:
	\begin{equation}
		R(\tau) = \frac{\sum_{t=1}^{N-\tau} (x_t - \overline{x}) (x_{t+\tau} - \overline{x})}{\sum_{t=1}^{N} (x_t - \overline{x})^2}
	\end{equation}
	Here range of $R(\tau) \in [-1,1]$ \\
	We can use it to find periodicity in Lorenz system. Furthermore, rapid decay of $R(\tau)$ indicates high chaos and slow decay of it refers predictability.\\
	\subsection{Homoclinic Orbit}
	Its a trajectory that starts and ends at the same saddle point.
	Which means it is said to be Homoclinic orbit if\\
	\begin{equation}
		\lim_{t\to -\infty} x(t) = x_0
	\end{equation}
	and
	\begin{equation}
		\lim_{t\to \infty} x(t) = x_0
	\end{equation}
	where $x_0$ is a saddle equilibrium.\\
	This types of orbits create a closed loop in phase space. Besides it is sensitive to perturbation as a result a tiny change destroy the orbit.
	\subsection{Heteroclinic Orbit}
	This types of trajectory connects two different saddle equilibria. It starts at a saddle point $x_a$ as $t\to-\infty$ and ends at another saddle $x_b$ as $t\to\infty$:
	\begin{equation}
		\lim_{t\to-\infty}x(t)= x_a	  and   \lim_{t\to\infty}x(t)= x_b
	\end{equation}
	It forms a bridge between two saddle points. However it is often seen in symmetric system.\\
	To find these types of orbits first we need to calculate the saddle equilibrias. we need to solve the values of $x$, $y$ and $z$ for $\dot{x}=0$, $\dot{y}=0$ and $\dot{z}=0$.\\
	after solving it we get,\\
	\begin{equation}
		(x,y,z) = (0,0,0)
	\end{equation}
	and
	\begin{equation}
		(x,y,z) = (\pm \sqrt{\beta(\rho -1)}, \pm \sqrt{\beta(\rho -1)}, \rho -1)
	\end{equation}
	\\
	\subsection{Work}
	Moreover, in order to find the work done on the particle firstly we calculate the expression for the acceleration by differentiating the the equation of this Lorenz system and for a unit mass particle we calculate the work on it.
	\begin{equation}
		a_x = \sigma(v_y - v_x);
		a_y = \gamma v_x + v_x z - v_z x +v_y;
		a_z = v_x y + v_y x - \beta v_z
	\end{equation} 
	using these expression for acceleration and for $m=1$ we compute the work by
	\begin{equation}
		W= \int \vec{F} \cdot d\vec{s}
	\end{equation}
	if the work done on it reaches to any constant value over time then we can say this particle would stop after a time. However we can also check if the work is positive or negative.
	\subsection{\label{sec:citeref}Citations and References}
	
	
	\subsubsection{Citations}
	
	
	\paragraph{Syntax}
	
	
	\paragraph{The options of the cite command itself}
	
	
	\subsubsection{Example citations}
	
	
	\subsubsection{References}
	
	
	\subsubsection{Example references}
	
	
	
	
	\subsection{Footnotes}%
	
	\section{Methodology}
	To find the geometric properties of the Lorenz system, firstly, we first take cross sections of the phase space as $p_x$ vs $x$, $p_y$ vs $p_z$ etc and observed them. We avoid terms like $p_x$ vs $y$. However, we also checked if the kinetic energy of the system was conserved or not. Since, for a system usually the total energy is conserved for a system, we therefore tried to calculate the work on a unit mass particle. After this, we summed and subtracted this work from the kinetic energy to find if the result was constant.
	we also tried to find some average values like average of $x$, $v_x$ etc. Besides, for a range of parameter value, homoclinic and heteroclinic orbits are found. We took the values of $\rho$, $\sigma$ and $\beta$ to be [0,30] as the parameters of the Lorenz system are positive. With step of 1, 29791 trajectories were examined for being homoclinic or heteroclinic or neither. As these types of trajectories are vary rare we expected to find a few of them.\\
	It is known that for a straight line the $\ddot{\vec{r}}$ is zero and for a circle $\dot{\vec{r}}$ is zero for this reason dot product and cross product of these terms were calculated over time. If these terms converge it might be a property of the system.\\
	For measuring Permutation entropy of a time series we choose $3$ points from that time series and sort them in increasing order to assign a permutation pattern with them. For 3 points, there are $3!$ possible patterns. After that, the probability each pattern was measured and used to calculate the permutation entropy. \\
	Shannon entropy and Renyi entropy were calculated using a different approach. Each time series was divided into a numbers of bins. For example, if the highest value and lowest value of a time series were $x_max$ and $x_min$ respectively, then $n$ bins were created between $x_{min}$ and $x_{max}$. We then counted how many points from the time series fell into each bin in order to determine the corresponding probabilities for each bin , which were then used to calculate the entropy. calculate entropy.
	
	\section{Results and Discussion}
	If we analysis the slices of phase space we find that each of the 2D sections has some kind of symmetry or anti-symmetry of the trajectory about a line. For different cuts the lines are different. We believe that if it were possible to let the time go for infinity, one might find that every location on a bounded region would be covered. 
	\FloatBarrier
	\begin{figure}[htbp]
		\centering
		\includegraphics[width=0.8\linewidth]{v_x_vs_x.png}
		\caption{Phase Space: $v_x$ vs $x$}
		\label{fig:vx_x}
	\end{figure}
	
	\begin{figure}[htbp]
		\centering
		\includegraphics[width=0.8\linewidth]{v_y_vs_y.png}
		\caption{Phase Space: $v_y$ vs $y$}
		\label{fig:vy_y}
	\end{figure}
	
	\begin{figure}[htbp]
		\centering
		\includegraphics[width=0.8\linewidth]{v_z_vs_z.png}
		\caption{Phase Space: $v_z$ vs $z$}
		\label{fig:vz_z}
	\end{figure}
	
	\begin{figure}[htbp]
		\centering
		\includegraphics[width=0.8\linewidth]{y_vs_x.png}
		\caption{Position Space: $y$ vs $x$}
		\label{fig:y_x}
	\end{figure}
	
	\begin{figure}[htbp]
		\centering
		\includegraphics[width=0.8\linewidth]{z_vs_y.png}
		\caption{Position Space: $z$ vs $y$}
		\label{fig:z_y}
	\end{figure}
	
	\begin{figure}[htbp]
		\centering
		\includegraphics[width=0.8\linewidth]{x_vs_z.png}
		\caption{Position Space: $x$ vs $z$}
		\label{fig:x_z}
	\end{figure}
	
	\begin{figure}[htbp]
		\centering
		\includegraphics[width=0.8\linewidth]{vy_vs_vx.png}
		\caption{Velocity Space: $v_y$ vs $v_x$}
		\label{fig:vy_vx}
	\end{figure}
	
	\begin{figure}[htbp]
		\centering
		\includegraphics[width=0.8\linewidth]{vz_vs_vy.png}
		\caption{Velocity Space: $v_z$ vs $v_y$}
		\label{fig:vz_vy}
	\end{figure}
	
	\begin{figure}[htbp]
		\centering
		\includegraphics[width=0.8\linewidth]{vx_vs_vz.png}
		\caption{Velocity Space: $v_x$ vs $v_z$}
		\label{fig:vx_vz}
	\end{figure}
	
		
	\FloatBarrier
	
	None of kinetic energy, work and their summation or subtraction are conserved. Kinetic energy change rapidly over time which can be guessed from the trajectory of a particle in this system as its velocity change with no known pattern. Furthermore at some time the value of kinetic energy reaches near zero. This phenomena indicates that the particle sometimes falls into the attractor. When we simulate the motion this can be seen. Besides work by the particle is always negative. Which means some external agent is there to put any object into this type of motion. 
	\FloatBarrier
	\begin{figure}[htbp]
		\centering
		\includegraphics[width=0.8\linewidth]{kinetic_energy_plot.png}
		\caption{Kinetic Energy of the particle for$m=1$}
		\label{fig:kinetic_energy}
	\end{figure}
	\begin{figure}[htbp]
		\centering
		\includegraphics[width=0.8\linewidth]{work.png}
		\caption{Work by the particle}
		\label{fig:work}
	\end{figure}
	
	\FloatBarrier
	However $\langle x\rangle$, $\langle y\rangle$ and  $\langle z\rangle$ converge to some values. which is due to their nature of orbiting around some attractor. \\
	With step=1 for the values of $\rho$, $\sigma$ and $\beta$ in the range [0,30] we have found some values of $(\sigma, \beta, \rho )$ for which we get the homoclinic orbits and hetero clinic orbits.
	\FloatBarrier
	\begin{figure}[htbp]
		\centering
		\includegraphics[width=1.5\linewidth]{homoclinic_heteroclinic_points.png}
		\caption{Parameters' value for Homoclinic and Heteroclinic orbits}
		\label{homoclini_heteroclinic}
	\end{figure}
	\FloatBarrier
	Neither the dot product nor the cross product of $\ddot{\vec{r}}$ and $\dot{\vec{r}}$ converged.\\
	Permutation entropy was measured by observing the pattern of 3 corresponding points. Initially, it was not stable, but after a period of time, this reached a constant value. As, at the beginning, the butterfly shape of the path is not obtained, we do not get the actual value of entropy. As we took three points, there are $3!$ patterns, and for this scenario the highest value to Permutation entropy one woukd get is $2.585$. But we did measure lower than that. If the trajectory was fully random we might get the highest value.
	\FloatBarrier
	\begin{figure}[htbp]
		\centering
		\includegraphics[width=0.8\linewidth]{PE_vs_time_x_y_z.png}
		\caption{Permutation Entropy}
		\label{PE_for_time_series}
	\end{figure}
	
	\FloatBarrier
	Similar behaviour was observed in the Shanon entropy's and Renyi entropy's graph. We chose  $\alpha = 2$ for calculate Renyi over time.
	\\
	for,
	\begin{itemize}
		\item \(\alpha < 1\): More sensitive to rare events
		\item \(\alpha = 1\): Renyi entropy $\to$ Shanon entropy
		\item \(\alpha > 2\): More focused on most frequent events
		\item \(\alpha \to \infty\): Minimum entropy
	\end{itemize}
	\FloatBarrier
	\begin{figure}[htbp]
		\centering
		\includegraphics[width=0.8\linewidth]{SE_vs_time_x_y_z.png}
		\caption{Shanon Entropy over time}
		\label{SE_forx_time_series}
	\end{figure}
	
	\begin{figure}[htbp]
		\centering
		\includegraphics[width=0.8\linewidth]{RE_vs_time_x_y_z.png}
		\caption{Renyi Entropy over time}
		\label{fig:Renyi Entropy}
	\end{figure}
	\begin{figure}[htbp]
		\centering
		\includegraphics[width=0.8\linewidth]{renyi_entropy_vs_alpha_for_100_bin.png}
		\caption{Renyi Entropy with respect to $\alpha$}
		\label{fig:Renyi Entropy_alpha}
	\end{figure}
	\FloatBarrier
	Meanwhile autocorrelation shows us $x(t)$ decays rapidly and faster than any other time series.
	Which means it is chaotic and has little long term memory as a result past value dont predict the future.
	However $y(t)$ also decay quickly but not rapidly like $x(t)$. It is also chaotic and confirms short predictability. 
	Where as $z(t)$ shows different story. Unlike $x$ and $y$ it oscillate more regularly.
	\FloatBarrier
	\begin{figure}[htbp]
		\centering
		\includegraphics[width=0.8\linewidth]{auto_correlation.png}
		\caption{Autocorrelation vs lag}
		\label{fig:autocorrelation}
	\end{figure}
	\FloatBarrier
	\section{Conclusion}
	In this work, we analysis Lorenz attractor from some different aspect to gain a deeper understanding of a complex dynamical system. Moving beyond some traditional measurements, we study this system with a view of information theory including Shannon entropy, Renyi entropy and Permutation entropy. In parallel, the computation of slides of the phase space, work, kinetic energy, parameters' value for homoclinic orbits and heteroclinic orbits provided a geometric interpretation of this system.\\
	Reaching to a azimuthal value as the temporal evolution of these entropies explain the usability of these quantities for a dynamical system. Crucially, we established a link between entropy measures and the structure of the system. Besides. the geometric study shows us there should exist some quantities which can be able to explain this system. The negative work implies the existance of attractive force and the randomness is kinetic energy tells us the random movement of particle in this system. Meanwhile the slides of phase space reveal the symmetry of them.\\ 
	While this study offers some valuable insights, we were unable to define any new entropy which would be perfect for any dynamical system. Although we started with a goal of defining a new complexity for dynamical systems, we still could not find any expression for this and future work could explore this part. 
	\begin{acknowledgments}
		We acknowledge helpful discussions with colleagues and support from our institutions. This work was supported by the XYZ Foundation (Grant No. 12345).
	\end{acknowledgments}
	
	\appendix
	\section{Technical Details}
	
	The appendix provides additional mathematical details omitted from the main text for readability.
	
	\begin{equation}
		\mathcal{R} = \sum_{i=1}^N \frac{x_i^2}{2\sigma^2}.
		\label{eq:appendix_eq}
	\end{equation}
	
	\bibliography{nonlinear}
	
\end{document}
% ****** End of file apssamp.tex ******